\documentclass[12pt]{article}
\usepackage{amsmath}
\usepackage{amssymb}
\usepackage{fancyhdr}
\usepackage{graphicx}
\usepackage{pdfpages}
\usepackage{listings}
\usepackage{color}
\usepackage{color}
\usepackage{lmodern}
\definecolor{dkgreen}{rgb}{0,0.6,0}
\definecolor{gray}{rgb}{0.5,0.5,0.5}
\definecolor{mauve}{rgb}{0.58,0,0.82}
\pagestyle{headings}
\setlength{\oddsidemargin}{0in}
\setlength{\evensidemargin}{0in}
\setlength{\textheight}{9in}
\setlength{\textwidth}{6.5in}
\setlength{\topmargin}{-0.5in}
\setlength{\headheight}{14pt}
\renewcommand*\rmdefault{lmss}
\renewcommand*\contentsname{Table of Contents}
\lstset{language=Matlab,
	aboveskip=3mm,
	belowskip=3mm,
	showstringspaces=false,
	basicstyle={\normalfont\ttfamily},
	numbers=left,
	numberstyle=\tiny\color{gray},
	keywordstyle=\color{blue},
	commentstyle=\color{dkgreen},
	stringstyle=\color{mauve},
	breaklines=true,
	breakatwhitespace=true,
	tabsize=4
}

\newcommand{\horizontalLine}{
	\begin{center}
		\hrule width 1.0\textwidth
	\end{center}
}

%%%%%%%%%%%%%%%%%%%%%%%%%%%%%%%%%%%%%%%%%%%%%%%%%%%%%%%%%%
%%%%%%%%%%%%%%%%%%%%%%%%%%%%%%%%%%%%%%%%%%%%%%%%%%%%%%%%%%
\title{\vspace{-3ex}\bf Final Project Report\\[2ex] 
       \normalsize ECS 171: Machine Learning\\Fall 2015}
\date{\today}
\author{\bf Team Name: The Muses\\ \bf William Otwell ()\\ \bf Marshall Hampson ()\\ \bf Nicholas Layton ()\\ \bf Steven Mackey ()\\ \bf Amos Too ()\\ \bf Michael Fiueroa ()\\ \bf Peggy Li ()}

%%%%%%%%%%%%%%%%%%%%%%%%%%%%%%%%%%%%%%%%%%%%%%%%%%%%%%%%%%
%%%%%%%%%%%%%%%%%%%%%%%%%%%%%%%%%%%%%%%%%%%%%%%%%%%%%%%%%%
\begin{document}
\maketitle
\pagebreak
\tableofcontents
\pagebreak

%%%%%%%%%%%%%%%%%%%%%%%%%%%%%%%%%%%%%%%%%%%%%%%%%%%%%%%%%%
%%%%%%%%%%%%%%%%%%%%%%%%%%%%%%%%%%%%%%%%%%%%%%%%%%%%%%%%%%
\section{Abstract}
\label{sec:abstract}
One-two sentences on what is the problem that you addressing. Two-three sentences
that contain a high-level description on how you address the problem (your method). 2-3 sentences
on the results. 1 sentence on what the advance that has been achieved with this work
can influence the field (what it will enable, for the specific field and whole area in general)

%%%%%%%%%%%%%%%%%%%%%%%%%%%%%%%%%%%%%%%%%%%%%%%%%%%%%%%%%%
%%%%%%%%%%%%%%%%%%%%%%%%%%%%%%%%%%%%%%%%%%%%%%%%%%%%%%%%%%
\horizontalLine
\section{Introduction}
\label{sec:introduction}
1-2 sentences of the general area, 1-2 sentences on the specific sub-area. A paragraph
on the specific problem and how it has been addressed so far. 1-2 sentences to a paragraph
on what is missing on current approaches and why this is important.
Then a paragraph on what this paper contributes - how you approach this problem and the
results of the approach - This paragraph should have \textbf{\textit{clear}}, \textbf{\textit{definite}} \textbf{\textit{claims}} on what you have
achieved in relation to what you claimed that is missing in the field (from the previous paragraph).
You should not add items that do not relate to the previous "missing/desired" advances
that you introduced before. If this happens, then either you have not introduced the challenges/missing/desired
in the previous paragraph adequately, or your claim is irrelevant to this
paper and has to be removed.
Some people include a last paragraph with the structure of the paper as the closing introduction
paragraph. This is up to you.

%%%%%%%%%%%%%%%%%%%%%%%%%%%%%%%%%%%%%%%%%%%%%%%%%%%%%%%%%%
%%%%%%%%%%%%%%%%%%%%%%%%%%%%%%%%%%%%%%%%%%%%%%%%%%%%%%%%%%
\horizontalLine
\section{Methods}
\label{sec:methods}
 Divide it into sections that are well-defined on the distinct components/algorithms/subproblems.
 Put references whenever you use/step on previously published work. Use sub-section
 indexing (3.1, 3.2, etc.). Describe fully your algorithms and methods, so that if anyone wants to
 reproduce your results can do so (sometimes this is difficult if there are many parameters, in
 which case a parameter file should be included as suppl. mat.).

\subsection{Linear Regression: Predicting Song Popularity}
\label{subsec:linearRegression}
blah blah blah

\subsection{Artificial Neural Network: ...Predicting what?...}
\label{subsec:ann}
blah blah blah

\subsection{K-Means Clustering: ...Clustering what?...}
\label{subsec:kMeans}
blah blah blah

%%%%%%%%%%%%%%%%%%%%%%%%%%%%%%%%%%%%%%%%%%%%%%%%%%%%%%%%%%
%%%%%%%%%%%%%%%%%%%%%%%%%%%%%%%%%%%%%%%%%%%%%%%%%%%%%%%%%%
\horizontalLine
\section{Results}
\label{sec:results}
This should only describe the results that you have obtained. Sometimes it make sense
to discuss your guess/opinion why this is happening, but this should be rare. This section is to
only present the facts about the method performance, robustness, complexity, etc.

\subsection{Analysis}
\label{subsec:analysis}
This is where you discuss everything that was presented in the Results section.
Why did the algorithm perform better in X and not in Y. It is ok to succinctly re-state strong
result claims (that were included in the previous section), as long as you continue to explain,
even speculate why this may be the case. For example: "Our algorithm was faster in X% of the
scenarios than what is currently available. The main drive behind this performance boost is the
Y module, which takes advantage of the Z characteristic of the problem. Indeed, ...."

\subsubsection{Linear Regression: Predicting Song Popularity}
\label{subsubsec:linearRegression}
blah blah blah

\subsubsection{Artificial Neural Network: ...Doing what?...}
\label{subsubsec:ann}
blah blah blah

\subsubsection{K-Means Clustering: ...Doing what?...}
\label{subsubsec:kMeans}
blah blah blah

\subsection{Conclusion}
\label{subsec:conclusion}
Finally, in what some papers refer to as "Conclusion", you provide 1-2 sentences of the purpose
of this report and 1-2 sentences of what was achieved (these 2-4 sentences are usually similar
or re-stating what was said in the abstract). Then you go on to discuss about what remains to
be done (the road ahead), why (the impact to the field) and how you think it can be achieved
(future work). You end the report with 1-2 sentences on how the work presented advances the
field in the grant scheme of things.

%%%%%%%%%%%%%%%%%%%%%%%%%%%%%%%%%%%%%%%%%%%%%%%%%%%%%%%%%%
%%%%%%%%%%%%%%%%%%%%%%%%%%%%%%%%%%%%%%%%%%%%%%%%%%%%%%%%%%
\horizontalLine
\section{References}
\label{sec:references}
 Please include all relevant references for the paper, to provide a succinct and accurate
 summary of past work and challenges in the field. Research has found that the more
 2
 articles you cite, the more you will be cited too. For this report you are expected to have 10-20
 references.

%%%%%%%%%%%%%%%%%%%%%%%%%%%%%%%%%%%%%%%%%%%%%%%%%%%%%%%%%%
%%%%%%%%%%%%%%%%%%%%%%%%%%%%%%%%%%%%%%%%%%%%%%%%%%%%%%%%%%
\horizontalLine
\section{Author Contributions}
\label{sec:authorContributions}

\begin{itemize}
    \item Methods:
    \begin{itemize}
        \item Linear Regression
        \begin{itemize}
            \item Peggy Li:
            \item Steven Mackey:
        \end{itemize}
        \item Artificial Neural Network
        \begin{itemize}
            \item Nicholas Layton:
            \item Michael Fiueroa:
        \end{itemize}
        \item K-Means Clustering
        \begin{itemize}
            \item William Otwell:
            \item Marshall Hampson:
            \item Amos Too:
        \end{itemize}
    \end{itemize}
\end{itemize}

%%%%%%%%%%%%%%%%%%%%%%%%%%%%%%%%%%%%%%%%%%%%%%%%%%%%%%%%%%
%%%%%%%%%%%%%%%%%%%%%%%%%%%%%%%%%%%%%%%%%%%%%%%%%%%%%%%%%%
\horizontalLine
\appendix
\section{Figures}
\label{sec:figures}

\end{document}

